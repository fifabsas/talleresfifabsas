\documentclass[a4paper]{article}
%\include{amsfonts}
\usepackage{amssymb,amsmath,amsthm,latexsym,epsfig,euscript,multicol}
% \usepackage{enumitem}
\usepackage{graphicx}
\usepackage{float}
\usepackage[margin=2cm]{geometry}
\usepackage{hyperref}

%\usepackage[utf8x]{inputenc}
\usepackage[spanish]{babel}% idioma castellano
\graphicspath{{fig/}}

% Caracteres especiales
\def\A{\mathbb{A}}
\def\C{\mathbb{C}}
\def \N{\mathbb{N}}
\def \P{\mathbb{P}}
\def \Q{\mathbb{Q}}
\def \R{\mathbb{R}}
\def \Z{\mathbb{Z}}

\def\zC{\mathbb{C}}
\def \zN{\mathbb{N}}

\def \zQ{\mathbb{Q}}
\def \zR{\mathbb{R}}
\def \zZ{\mathbb{Z}}

%  Ejercicio:
\newtheorem{ejer}{Unidad}
\newcommand{\bej}{\begin{ejer} \rm}
\newcommand{\fej}{\end{ejer}}


%
\def\d{\displaystyle}

%Encabezado y pie de página
\usepackage{fancyhdr} % Headers and footers
\setlength{\headwidth}{16.5cm}
\pagestyle{fancy} % All pages have headers and footers


\renewcommand{\headrulewidth}{1.5pt}


\renewcommand{\footrulewidth}{1.5pt}

% \fancyhead{} % Blank out the default header
% \fancyfoot{} % Blank out the default footer
\fancyhead[L]{Ejercicios complementarios del taller de Python}
% \fancyhead[R]{{\small Departamento de Física, FCEN, UBA}}% Custom header text
\fancyfoot[R]{\thepage} % Custom footer text
\fancyfoot[L]{FIFA}


%\topmargin-2cm \vsize 29.5cm \hsize 21cm
%\setlength{\textwidth}{16.5cm} \setlength{\textheight}{23.5cm}
%\setlength{\oddsidemargin}{0.0cm}
%\setlength{\evensidemargin}{0.0cm}

%\linespread{1.4cm}
\usepackage{setspace}
 \onehalfspacing

%-------------------------------------------------------------------------------------------------
%-------------------------------COMIENZA EL TEXTO-------------------------------------------------
%-------------------------------------------------------------------------------------------------
\begin{document}
\thispagestyle{empty}

% \vskip 1cm

% \centerline{{\small \textit{Departamento de Física}}}
% \centerline{{\small \textit{Facultad de Ciencias Exactas y Naturales }}}
% \centerline{{\small \textit{Universidad de Buenos Aires}}}
% \vskip 1cm

\centerline{{\bf\Large{Ejercicios complementarios del taller de Python}}}

\centerline{{\ttfamily Talleres FIFA (Federación Interestudiantil de Físicos de Argentina)}}

\begin{figure}[H]
 \centering
   \includegraphics[width=0.3\textwidth]{logos_python_fifa.png}
 \label{FIFA}
 \end{figure}

\bigskip


\textbf{Observación}

En caso de que el IDE de sus computadoras no funcione, se puede probar programar online en la web: \url{https://www.python.org/shell/}
%https://cloud.sagemath.com/#settings
%https://www.pythonanywhere.com

\bigskip
\centerline{\bf  Ejercicios propuestos}
\bigskip

\begin{enumerate}
	\item \texttt{es\_primo($n$)}: devuelve True si el input es un número primo. False en caso contrario.
	\item \texttt{factorial($n$)}: devuelve el factorial del input $n$. Recordar que el factorial se define: $n! = n\cdot(n-1)\cdot(n-2)\cdot \,\ldots\, \cdot1$
	\item \texttt{absoluto($x$)}: devuelve el valor absoluto del input $x$. Si $x<0$ debería devolver $-x$ y si $x\ge 0$ debería devolver $x$.
	\item \texttt{norma(x, y, z)}: devuelve la distancia euclídea entre el punto elegido y el origen de coordenadas
		\begin{itemize}
			\item modifique la función para que reciba un único parámetro v, que sea una tupla o lista de largo 3.
			\item extienda esta función para que reciba dos puntos y calcule la norma del vector que los une.
			\item puede extender la función para que calcule otras normas no euclídeas.
		\end{itemize}
	\item \texttt{perp(v1, v2)}: devuelve True si los vectores son perpendiculares y False si no lo son (recordar que es fácil ver si dos vectores son perpendiculares viendo su producto interno: $\vec{v}\cdot \vec{w} = v_1w_1 + v_2w_2 + v_3 w_3$. )
		\begin{itemize}
			\item Generalice a dos vectores de largo $N$, con un $N$ arbitrario.
		\end{itemize}
	\item \texttt{prod\_vectorial(v, w)} devuelve el producto vectorial $\vec{v}\times \vec{w} = (v_2 w_3 - v_3 w_2; v_3 w_1 - v_1 w_3; v_1 w_2 - v_2 w_1)$. Para verificar la función pueden verificar que $(1;0;0)\times (0;1;0) = (0;0;1)$ (esto es la regla de la mano derecha: $\hat{i} \times \hat{j} = \hat{k}$, para los que la conozcan).
	\item Considere la serie de Taylor de la función exponencial $f(x)=e^x=\sum\limits_{n=0}^{\infty} (x^n) / n!$. Como esta suma es infinita se imaginarán que tarda infinito tiempo en converger al valor con error nulo. No obstante, cada término que se calcula acerca la estimación al valor real. El objetivo de este ejercicio es ver la convergencia de esta serie, para realizarlo les conviene tener algunas de las funciones de los ejercicios anteriores a mano.\\
		Escriba una función \texttt{exp\_taylor($x$, $N$)} que reciba como parámetros el valor de $x$ donde evaluar la exponencial y $N$ siendo el orden de precisión deseado (es decir que la sumatoria en vez de ser hasta $\infty$ será hasta $N$) y devuelva el resultado de la sumatoria parcial obtenida.\\
		Luego compare con el valor "real" de la función (que pueden obtener, por ahora, usando una calculadora o Google o Mathematica. La próxima clase veremos de donde podemos obtener el valor desde python). Por ejemplo: Se quiere ver la convergencia para $x=4$, entonces buscamos el valor posta $e^4 \approx 54.5981500331$, y para ver el error hacemos: $\textup{error} = |\,54.5981500331 - \texttt{exp\_taylor(4, N)}\,|$.
		La pregunta que se quiere saber es, a qué orden necesito hacer la cuenta para encontrar un que difiera del real en menos de $0.1$. Y $0.01$? $0.001$? etc\ldots.\\
		Esto va a estar bueno cuando aprendamos a hacer gráficos, vamos a poder ver el error en función del orden de aproximación.

	\item \texttt{verificar\_nombre(str)}: Esta función va a recibir un string, y va a verificar si es compatible como nombre de archivo devolviendo True o False. Vamos a aceptar nombres solamente si \emph{no} contienen los caracteres: [ ] ( ) \{ \} , / ; . *

	\item ¡Adivina un número del uno al diez! Defina un número del 1 al 10, luego pídale al usuario que ingrese un número. Si adivina el número que muestre un mensaje de felicitaciones y si no que siga intentando.\\
		Para esto puede usar la función \texttt{input}. Deberá buscar en internet o la documentación de esta función, en particular para ver qué tipo de dato devuelve esta función (si es un \texttt{int, str, \ldots}).
\end{enumerate}

\bej \textbf{Comandos y variables}

Las variables son formas de almacenar información. La diversidad de esta información da origen a la diversidad de tipos de variables. Los comandos son operaciones que se pueden hacer sobre las variables para transformarlas, leerlas u obtener otras cosas de ellas. Recomendamos utilizar el comando \textit{type(variable)} al final de cada item para que Python nos devuelva el tipo de variable con la que estamos trabajando.

\begin{itemize}

\item Comandos \textit{print} e \textit{input}

Haga que Python realice las siguientes tareas:

\begin{enumerate}
 \item Escriba el texto 'Hola mundo'.
 \item Deje un espacio de tabulación o un salto de línea entre \textit{hola} y \textit{mundo} (pruebe con $\backslash t$ y $\backslash n$).

 \item Guarde en una variable el texto 'Me tiemblan los dedos de la emoción de saber que estoy programando'.
 \item Guarde en dos variables dos textos, y que imprima ambas variables separadas por un espacio de tabulación. (¿Se puede usar el + para eso?)
 \item Pruebe escribir a=int(input('Dale un valor a \textit{a}: ')) ¿Qué tipo de variable es $a$? ¿Por qué sería necesario el comando \textit{int}? ¿Cambia la función del + con el tipo de variable que vincula?
 \item Aplique el comando \textit{len(variable)} sobre una de las variables con texto y vea qué devuelve.
\end{enumerate}


\item Variables \textit{integer} y \textit{float}

¿Qué diferencia hay entre ambos tipos de variables?

\begin{enumerate}
 \item Guarde en una variable \textit{a} el valor 7, en otra variable \textit{b} el valor 9, y que los sume.
 \item Pruebe ahora restarlas, multiplicarlas y dividirlas. ¿Qué otras operaciones matemáticas se pueden hacer con Python?
 \item Guarde en una variable el resultado de la división de dos variables \textit{integer}. ¿Qué tipo de variable es este resultado?
 \item Averigue: ¿Cuántas cifras puede almacenar una variable tipo \textit{float}? \footnote{Pista: \url{https://docs.python.org/3/tutorial/floatingpoint.html}}
 \item ¿Qué tipo de variable devuelve Python al aplicar \textit{round}(x,n)? (siendo \textit{x} un \textit{float} y \textit{n} la cantidad de decimales que uno desea redondear).
\end{enumerate}


\item Variables \textit{boolean}

\begin{enumerate}
 \item ¿Qué valor obtendrá de $5>5$? ¿Y de $5>=5$?
 \item ¿Qué obtendrá de escribir $ 4==5 == 3 !=2$? ¿Son útiles los paréntesis?
 \item ¿Valen los operadores $==$ y $!=$ para \textit{strings}?
\end{enumerate}

\item Variables tipo \textit{list} y tipo \textit{tuple}

¿Qué diferencia hay entre ambos tipos de variables?

\begin{enumerate}
 \item Arme una \textit{lista} con las materias en las que debe el examen final y asígnela a una variable.
 \item Pasó el cuatrimestre y nos colgamos todos. Agregá un elemento a la lista anterior (\textit{append}) y pedile que te muestre el segundo elemento.
 \item ¿Cuál es la diferencia entre \textit{extend} y \textit{append}? ¿Y con  \textit{insert}?
 \item ¿Qué diferencia hay entre los comandos \textit{remove} y \textit{pop}? ¿Qué hace el comando \textit{sort}?
 \item Armada una lista, aplíquele la función \textit{len}.
 \item Pruebe \textit{range}(10), \textit{range}(3,13), \textit{range}(2,26,2). ¿Qué tipo de variable es el resultado?
 \item Reconozca algunos de estos métodos en la descripción de las listas usando \textit{help(list)} (para ver más en el \textit{help} presionar \textit{enter} y para salir \textit{q}).
\end{enumerate}
\end{itemize}
\fej

  \bigskip

\bej \textbf{Bibliotecas} (o \textbf{libraries} en inglés)

Las bibliotecas contienen funciones pensadas y distribuidas para extender las posibilidades de Python.
Lo primero que debemos hacer es cargar la biblioteca con el comando \textit{import}.
Para resolver este ejercicio usaremos la biblioteca \textit{math}.

Probar luego de eso resolver los siguientes ejercicios.

\begin{enumerate}
 \item Calcular el seno de $\pi$ en radianes. Si no te gusta, cambialo a grados (funciones \textit{degrees} o \textit{radians})
 \item Calcular  $arctan(1/2)$.
 \item Calcular $2^{3^{4}}$ y $e^{\pi}$
 \item Calcular $\log_3(25)$.
 \item Calcular $e^{ln(x)}$ siendo x el número que quiera.
\end{enumerate}
\fej

\bigskip

\textbf{Pregunta aislada:} ¿Qué pasa si se aplica la función \textit{type} sobre la respuesta a \textit{type} sobre una variable?

\bigskip

\bej \textbf{Condicionales y funciones}

Una función es un paquete aislado e independiente de acciones a realizar. Pueden ser definidas y nunca utilizadas. Definir funciones como serie de acciones a realizar con una o más variables nos da la posibilidad de aplicar ese paquete más de una vez sin necesidad de repetir código.

Implemente las siguientes funciones (\textit{def} o \textit{lambda}) en Python:

\begin{enumerate}
 \item \textbf{doble}(n): muestre el doble de n.
 \item \textbf{promedio3}(n1, n2, n3): calcule y muestre el promedio de tres números pero redondeando para arriba. %Usar ceil(x) y floor(x)
 \item \textbf{iguales}(n1,n2): devuelve True si n1=n2.
 \item \textbf{divisible}(n,d): devuelve True si n es divisible por d.
 \item \textbf{factorial}(n): devuelve el valor factorial de n.
 \item \textbf{primo}(n): devuelve True si n es un número primo.
 \item \textbf{norma}(x,y,z): devuelve la distancia de un punto al cero de un espacio euclidiano.
\end{enumerate}
\fej

\bigskip

\bej \textbf{Gráficos}

Para encaminarnos al próximo encuentro, resulta básico y muy útil saber graficar funciones de una variable respecto a otra. Para esto hay funciones ya armadas en las bibliotecas \textit{Numpy} y \textit{Matplotlib}.

Para experimentos numéricos, procederemos siempre de la misma manera:

\begin{enumerate}
 \item Crearemos un dominio de valores equiespaciados con \textit{numpy.linspace()}.
 \item Crearemos una imagen asociada a una función $f(x)$ definida por nosotros.
 \item Aportaremos ruido a los datos, simulando "mediciones", con funciones como \textit{numpy.random.rand()}
 \item Graficaremos $x$ vs $y$ mediante las funciones \textit{scatter} o \textit{plot}, creando una grilla, dándole límites al dominio y la imagen mostrados y dándole nombres tanto a los ejes como al gráfico.
\end{enumerate}

Interesante sería ver cómo hacer varios gráficos juntos (\textit{subplot}) o cómo hacer gráficos de tipo histograma, cosas que abarcaremos en el próximo encuentro.

\fej

\bigskip

\bej \textbf{Juegos en Python}

Esta sección tiene un par de desafíos de aplicación de lo visto hasta ahora. Son totalmente resolubles a esta altura del aprendizaje.

Intente crear los siguientes juegos en Python:
\begin{enumerate}
 \item ¡Adivina un número del uno al diez! (que el usuario pruebe números del uno al diez y el programa le diga si adivinó o no)
 \item Piedra, papel o tijera (Variante para valientes: Piedra, papel, tijera, lagarto y spock).
 \item Ta-te-tí.
\end{enumerate}
\fej

\begin{figure}[H]
 \centering
 \includegraphics[width=0.4\textwidth]{einstein.jpeg}
 \label{Einstein}
 \end{figure}


\end{document}
