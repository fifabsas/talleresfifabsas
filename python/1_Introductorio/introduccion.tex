
% Default to the notebook output style

    


% Inherit from the specified cell style.




    
\documentclass{article}

    
    
    \usepackage{graphicx} % Used to insert images
    \usepackage{adjustbox} % Used to constrain images to a maximum size 
    \usepackage{color} % Allow colors to be defined
    \usepackage{enumerate} % Needed for markdown enumerations to work
    \usepackage{geometry} % Used to adjust the document margins
    \usepackage{amsmath} % Equations
    \usepackage{amssymb} % Equations
    \usepackage{eurosym} % defines \euro
    \usepackage[mathletters]{ucs} % Extended unicode (utf-8) support
    \usepackage[utf8x]{inputenc} % Allow utf-8 characters in the tex document
    \usepackage{fancyvrb} % verbatim replacement that allows latex
    \usepackage{grffile} % extends the file name processing of package graphics 
                         % to support a larger range 
    % The hyperref package gives us a pdf with properly built
    % internal navigation ('pdf bookmarks' for the table of contents,
    % internal cross-reference links, web links for URLs, etc.)
    \usepackage{hyperref}
    \usepackage{longtable} % longtable support required by pandoc >1.10
    \usepackage{booktabs}  % table support for pandoc > 1.12.2
    \usepackage{ulem} % ulem is needed to support strikethroughs (\sout)
    

    
    
    \definecolor{orange}{cmyk}{0,0.4,0.8,0.2}
    \definecolor{darkorange}{rgb}{.71,0.21,0.01}
    \definecolor{darkgreen}{rgb}{.12,.54,.11}
    \definecolor{myteal}{rgb}{.26, .44, .56}
    \definecolor{gray}{gray}{0.45}
    \definecolor{lightgray}{gray}{.95}
    \definecolor{mediumgray}{gray}{.8}
    \definecolor{inputbackground}{rgb}{.95, .95, .85}
    \definecolor{outputbackground}{rgb}{.95, .95, .95}
    \definecolor{traceback}{rgb}{1, .95, .95}
    % ansi colors
    \definecolor{red}{rgb}{.6,0,0}
    \definecolor{green}{rgb}{0,.65,0}
    \definecolor{brown}{rgb}{0.6,0.6,0}
    \definecolor{blue}{rgb}{0,.145,.698}
    \definecolor{purple}{rgb}{.698,.145,.698}
    \definecolor{cyan}{rgb}{0,.698,.698}
    \definecolor{lightgray}{gray}{0.5}
    
    % bright ansi colors
    \definecolor{darkgray}{gray}{0.25}
    \definecolor{lightred}{rgb}{1.0,0.39,0.28}
    \definecolor{lightgreen}{rgb}{0.48,0.99,0.0}
    \definecolor{lightblue}{rgb}{0.53,0.81,0.92}
    \definecolor{lightpurple}{rgb}{0.87,0.63,0.87}
    \definecolor{lightcyan}{rgb}{0.5,1.0,0.83}
    
    % commands and environments needed by pandoc snippets
    % extracted from the output of `pandoc -s`
    \providecommand{\tightlist}{%
      \setlength{\itemsep}{0pt}\setlength{\parskip}{0pt}}
    \DefineVerbatimEnvironment{Highlighting}{Verbatim}{commandchars=\\\{\}}
    % Add ',fontsize=\small' for more characters per line
    \newenvironment{Shaded}{}{}
    \newcommand{\KeywordTok}[1]{\textcolor[rgb]{0.00,0.44,0.13}{\textbf{{#1}}}}
    \newcommand{\DataTypeTok}[1]{\textcolor[rgb]{0.56,0.13,0.00}{{#1}}}
    \newcommand{\DecValTok}[1]{\textcolor[rgb]{0.25,0.63,0.44}{{#1}}}
    \newcommand{\BaseNTok}[1]{\textcolor[rgb]{0.25,0.63,0.44}{{#1}}}
    \newcommand{\FloatTok}[1]{\textcolor[rgb]{0.25,0.63,0.44}{{#1}}}
    \newcommand{\CharTok}[1]{\textcolor[rgb]{0.25,0.44,0.63}{{#1}}}
    \newcommand{\StringTok}[1]{\textcolor[rgb]{0.25,0.44,0.63}{{#1}}}
    \newcommand{\CommentTok}[1]{\textcolor[rgb]{0.38,0.63,0.69}{\textit{{#1}}}}
    \newcommand{\OtherTok}[1]{\textcolor[rgb]{0.00,0.44,0.13}{{#1}}}
    \newcommand{\AlertTok}[1]{\textcolor[rgb]{1.00,0.00,0.00}{\textbf{{#1}}}}
    \newcommand{\FunctionTok}[1]{\textcolor[rgb]{0.02,0.16,0.49}{{#1}}}
    \newcommand{\RegionMarkerTok}[1]{{#1}}
    \newcommand{\ErrorTok}[1]{\textcolor[rgb]{1.00,0.00,0.00}{\textbf{{#1}}}}
    \newcommand{\NormalTok}[1]{{#1}}
    
    % Additional commands for more recent versions of Pandoc
    \newcommand{\ConstantTok}[1]{\textcolor[rgb]{0.53,0.00,0.00}{{#1}}}
    \newcommand{\SpecialCharTok}[1]{\textcolor[rgb]{0.25,0.44,0.63}{{#1}}}
    \newcommand{\VerbatimStringTok}[1]{\textcolor[rgb]{0.25,0.44,0.63}{{#1}}}
    \newcommand{\SpecialStringTok}[1]{\textcolor[rgb]{0.73,0.40,0.53}{{#1}}}
    \newcommand{\ImportTok}[1]{{#1}}
    \newcommand{\DocumentationTok}[1]{\textcolor[rgb]{0.73,0.13,0.13}{\textit{{#1}}}}
    \newcommand{\AnnotationTok}[1]{\textcolor[rgb]{0.38,0.63,0.69}{\textbf{\textit{{#1}}}}}
    \newcommand{\CommentVarTok}[1]{\textcolor[rgb]{0.38,0.63,0.69}{\textbf{\textit{{#1}}}}}
    \newcommand{\VariableTok}[1]{\textcolor[rgb]{0.10,0.09,0.49}{{#1}}}
    \newcommand{\ControlFlowTok}[1]{\textcolor[rgb]{0.00,0.44,0.13}{\textbf{{#1}}}}
    \newcommand{\OperatorTok}[1]{\textcolor[rgb]{0.40,0.40,0.40}{{#1}}}
    \newcommand{\BuiltInTok}[1]{{#1}}
    \newcommand{\ExtensionTok}[1]{{#1}}
    \newcommand{\PreprocessorTok}[1]{\textcolor[rgb]{0.74,0.48,0.00}{{#1}}}
    \newcommand{\AttributeTok}[1]{\textcolor[rgb]{0.49,0.56,0.16}{{#1}}}
    \newcommand{\InformationTok}[1]{\textcolor[rgb]{0.38,0.63,0.69}{\textbf{\textit{{#1}}}}}
    \newcommand{\WarningTok}[1]{\textcolor[rgb]{0.38,0.63,0.69}{\textbf{\textit{{#1}}}}}
    
    
    % Define a nice break command that doesn't care if a line doesn't already
    % exist.
    \def\br{\hspace*{\fill} \\* }
    % Math Jax compatability definitions
    \def\gt{>}
    \def\lt{<}
    % Document parameters
    \title{introduccion}
    
    
    

    % Pygments definitions
    
\makeatletter
\def\PY@reset{\let\PY@it=\relax \let\PY@bf=\relax%
    \let\PY@ul=\relax \let\PY@tc=\relax%
    \let\PY@bc=\relax \let\PY@ff=\relax}
\def\PY@tok#1{\csname PY@tok@#1\endcsname}
\def\PY@toks#1+{\ifx\relax#1\empty\else%
    \PY@tok{#1}\expandafter\PY@toks\fi}
\def\PY@do#1{\PY@bc{\PY@tc{\PY@ul{%
    \PY@it{\PY@bf{\PY@ff{#1}}}}}}}
\def\PY#1#2{\PY@reset\PY@toks#1+\relax+\PY@do{#2}}

\expandafter\def\csname PY@tok@gd\endcsname{\def\PY@tc##1{\textcolor[rgb]{0.63,0.00,0.00}{##1}}}
\expandafter\def\csname PY@tok@gu\endcsname{\let\PY@bf=\textbf\def\PY@tc##1{\textcolor[rgb]{0.50,0.00,0.50}{##1}}}
\expandafter\def\csname PY@tok@gt\endcsname{\def\PY@tc##1{\textcolor[rgb]{0.00,0.27,0.87}{##1}}}
\expandafter\def\csname PY@tok@gs\endcsname{\let\PY@bf=\textbf}
\expandafter\def\csname PY@tok@gr\endcsname{\def\PY@tc##1{\textcolor[rgb]{1.00,0.00,0.00}{##1}}}
\expandafter\def\csname PY@tok@cm\endcsname{\let\PY@it=\textit\def\PY@tc##1{\textcolor[rgb]{0.25,0.50,0.50}{##1}}}
\expandafter\def\csname PY@tok@vg\endcsname{\def\PY@tc##1{\textcolor[rgb]{0.10,0.09,0.49}{##1}}}
\expandafter\def\csname PY@tok@m\endcsname{\def\PY@tc##1{\textcolor[rgb]{0.40,0.40,0.40}{##1}}}
\expandafter\def\csname PY@tok@mh\endcsname{\def\PY@tc##1{\textcolor[rgb]{0.40,0.40,0.40}{##1}}}
\expandafter\def\csname PY@tok@go\endcsname{\def\PY@tc##1{\textcolor[rgb]{0.53,0.53,0.53}{##1}}}
\expandafter\def\csname PY@tok@ge\endcsname{\let\PY@it=\textit}
\expandafter\def\csname PY@tok@vc\endcsname{\def\PY@tc##1{\textcolor[rgb]{0.10,0.09,0.49}{##1}}}
\expandafter\def\csname PY@tok@il\endcsname{\def\PY@tc##1{\textcolor[rgb]{0.40,0.40,0.40}{##1}}}
\expandafter\def\csname PY@tok@cs\endcsname{\let\PY@it=\textit\def\PY@tc##1{\textcolor[rgb]{0.25,0.50,0.50}{##1}}}
\expandafter\def\csname PY@tok@cp\endcsname{\def\PY@tc##1{\textcolor[rgb]{0.74,0.48,0.00}{##1}}}
\expandafter\def\csname PY@tok@gi\endcsname{\def\PY@tc##1{\textcolor[rgb]{0.00,0.63,0.00}{##1}}}
\expandafter\def\csname PY@tok@gh\endcsname{\let\PY@bf=\textbf\def\PY@tc##1{\textcolor[rgb]{0.00,0.00,0.50}{##1}}}
\expandafter\def\csname PY@tok@ni\endcsname{\let\PY@bf=\textbf\def\PY@tc##1{\textcolor[rgb]{0.60,0.60,0.60}{##1}}}
\expandafter\def\csname PY@tok@nl\endcsname{\def\PY@tc##1{\textcolor[rgb]{0.63,0.63,0.00}{##1}}}
\expandafter\def\csname PY@tok@nn\endcsname{\let\PY@bf=\textbf\def\PY@tc##1{\textcolor[rgb]{0.00,0.00,1.00}{##1}}}
\expandafter\def\csname PY@tok@no\endcsname{\def\PY@tc##1{\textcolor[rgb]{0.53,0.00,0.00}{##1}}}
\expandafter\def\csname PY@tok@na\endcsname{\def\PY@tc##1{\textcolor[rgb]{0.49,0.56,0.16}{##1}}}
\expandafter\def\csname PY@tok@nb\endcsname{\def\PY@tc##1{\textcolor[rgb]{0.00,0.50,0.00}{##1}}}
\expandafter\def\csname PY@tok@nc\endcsname{\let\PY@bf=\textbf\def\PY@tc##1{\textcolor[rgb]{0.00,0.00,1.00}{##1}}}
\expandafter\def\csname PY@tok@nd\endcsname{\def\PY@tc##1{\textcolor[rgb]{0.67,0.13,1.00}{##1}}}
\expandafter\def\csname PY@tok@ne\endcsname{\let\PY@bf=\textbf\def\PY@tc##1{\textcolor[rgb]{0.82,0.25,0.23}{##1}}}
\expandafter\def\csname PY@tok@nf\endcsname{\def\PY@tc##1{\textcolor[rgb]{0.00,0.00,1.00}{##1}}}
\expandafter\def\csname PY@tok@si\endcsname{\let\PY@bf=\textbf\def\PY@tc##1{\textcolor[rgb]{0.73,0.40,0.53}{##1}}}
\expandafter\def\csname PY@tok@s2\endcsname{\def\PY@tc##1{\textcolor[rgb]{0.73,0.13,0.13}{##1}}}
\expandafter\def\csname PY@tok@vi\endcsname{\def\PY@tc##1{\textcolor[rgb]{0.10,0.09,0.49}{##1}}}
\expandafter\def\csname PY@tok@nt\endcsname{\let\PY@bf=\textbf\def\PY@tc##1{\textcolor[rgb]{0.00,0.50,0.00}{##1}}}
\expandafter\def\csname PY@tok@nv\endcsname{\def\PY@tc##1{\textcolor[rgb]{0.10,0.09,0.49}{##1}}}
\expandafter\def\csname PY@tok@s1\endcsname{\def\PY@tc##1{\textcolor[rgb]{0.73,0.13,0.13}{##1}}}
\expandafter\def\csname PY@tok@sh\endcsname{\def\PY@tc##1{\textcolor[rgb]{0.73,0.13,0.13}{##1}}}
\expandafter\def\csname PY@tok@sc\endcsname{\def\PY@tc##1{\textcolor[rgb]{0.73,0.13,0.13}{##1}}}
\expandafter\def\csname PY@tok@sx\endcsname{\def\PY@tc##1{\textcolor[rgb]{0.00,0.50,0.00}{##1}}}
\expandafter\def\csname PY@tok@bp\endcsname{\def\PY@tc##1{\textcolor[rgb]{0.00,0.50,0.00}{##1}}}
\expandafter\def\csname PY@tok@c1\endcsname{\let\PY@it=\textit\def\PY@tc##1{\textcolor[rgb]{0.25,0.50,0.50}{##1}}}
\expandafter\def\csname PY@tok@kc\endcsname{\let\PY@bf=\textbf\def\PY@tc##1{\textcolor[rgb]{0.00,0.50,0.00}{##1}}}
\expandafter\def\csname PY@tok@c\endcsname{\let\PY@it=\textit\def\PY@tc##1{\textcolor[rgb]{0.25,0.50,0.50}{##1}}}
\expandafter\def\csname PY@tok@mf\endcsname{\def\PY@tc##1{\textcolor[rgb]{0.40,0.40,0.40}{##1}}}
\expandafter\def\csname PY@tok@err\endcsname{\def\PY@bc##1{\setlength{\fboxsep}{0pt}\fcolorbox[rgb]{1.00,0.00,0.00}{1,1,1}{\strut ##1}}}
\expandafter\def\csname PY@tok@kd\endcsname{\let\PY@bf=\textbf\def\PY@tc##1{\textcolor[rgb]{0.00,0.50,0.00}{##1}}}
\expandafter\def\csname PY@tok@ss\endcsname{\def\PY@tc##1{\textcolor[rgb]{0.10,0.09,0.49}{##1}}}
\expandafter\def\csname PY@tok@sr\endcsname{\def\PY@tc##1{\textcolor[rgb]{0.73,0.40,0.53}{##1}}}
\expandafter\def\csname PY@tok@mo\endcsname{\def\PY@tc##1{\textcolor[rgb]{0.40,0.40,0.40}{##1}}}
\expandafter\def\csname PY@tok@kn\endcsname{\let\PY@bf=\textbf\def\PY@tc##1{\textcolor[rgb]{0.00,0.50,0.00}{##1}}}
\expandafter\def\csname PY@tok@mi\endcsname{\def\PY@tc##1{\textcolor[rgb]{0.40,0.40,0.40}{##1}}}
\expandafter\def\csname PY@tok@gp\endcsname{\let\PY@bf=\textbf\def\PY@tc##1{\textcolor[rgb]{0.00,0.00,0.50}{##1}}}
\expandafter\def\csname PY@tok@o\endcsname{\def\PY@tc##1{\textcolor[rgb]{0.40,0.40,0.40}{##1}}}
\expandafter\def\csname PY@tok@kr\endcsname{\let\PY@bf=\textbf\def\PY@tc##1{\textcolor[rgb]{0.00,0.50,0.00}{##1}}}
\expandafter\def\csname PY@tok@s\endcsname{\def\PY@tc##1{\textcolor[rgb]{0.73,0.13,0.13}{##1}}}
\expandafter\def\csname PY@tok@kp\endcsname{\def\PY@tc##1{\textcolor[rgb]{0.00,0.50,0.00}{##1}}}
\expandafter\def\csname PY@tok@w\endcsname{\def\PY@tc##1{\textcolor[rgb]{0.73,0.73,0.73}{##1}}}
\expandafter\def\csname PY@tok@kt\endcsname{\def\PY@tc##1{\textcolor[rgb]{0.69,0.00,0.25}{##1}}}
\expandafter\def\csname PY@tok@ow\endcsname{\let\PY@bf=\textbf\def\PY@tc##1{\textcolor[rgb]{0.67,0.13,1.00}{##1}}}
\expandafter\def\csname PY@tok@sb\endcsname{\def\PY@tc##1{\textcolor[rgb]{0.73,0.13,0.13}{##1}}}
\expandafter\def\csname PY@tok@k\endcsname{\let\PY@bf=\textbf\def\PY@tc##1{\textcolor[rgb]{0.00,0.50,0.00}{##1}}}
\expandafter\def\csname PY@tok@se\endcsname{\let\PY@bf=\textbf\def\PY@tc##1{\textcolor[rgb]{0.73,0.40,0.13}{##1}}}
\expandafter\def\csname PY@tok@sd\endcsname{\let\PY@it=\textit\def\PY@tc##1{\textcolor[rgb]{0.73,0.13,0.13}{##1}}}

\def\PYZbs{\char`\\}
\def\PYZus{\char`\_}
\def\PYZob{\char`\{}
\def\PYZcb{\char`\}}
\def\PYZca{\char`\^}
\def\PYZam{\char`\&}
\def\PYZlt{\char`\<}
\def\PYZgt{\char`\>}
\def\PYZsh{\char`\#}
\def\PYZpc{\char`\%}
\def\PYZdl{\char`\$}
\def\PYZhy{\char`\-}
\def\PYZsq{\char`\'}
\def\PYZdq{\char`\"}
\def\PYZti{\char`\~}
% for compatibility with earlier versions
\def\PYZat{@}
\def\PYZlb{[}
\def\PYZrb{]}
\makeatother


    % Exact colors from NB
    \definecolor{incolor}{rgb}{0.0, 0.0, 0.5}
    \definecolor{outcolor}{rgb}{0.545, 0.0, 0.0}



    
    % Prevent overflowing lines due to hard-to-break entities
    \sloppy 
    % Setup hyperref package
    \hypersetup{
      breaklinks=true,  % so long urls are correctly broken across lines
      colorlinks=true,
      urlcolor=blue,
      linkcolor=darkorange,
      citecolor=darkgreen,
      }
    % Slightly bigger margins than the latex defaults
    
    \geometry{verbose,tmargin=1in,bmargin=1in,lmargin=1in,rmargin=1in}
    
    

    \begin{document}
    
    
    \maketitle
    
    

    
    \begin{figure}[htbp]
\centering
\includegraphics{logos_python_fifa.png}
\caption{taller\_python}
\end{figure}

    \section{Programar ¿con qué se
come?}\label{programar-con-quuxe9-se-come}

    La computadora es una gran gran calculadora que permite hacer cualquier
tipo de cuenta de las que necesitemos dentro de la Física (y de la vida
también) mientras sepamos cómo decirle a la máquina qué cómputos hacer.

La computadora para hacer cuentas tiene que almacenar los números que
necesitemos y luego hacer operaciones con ellos. Nuestros valores
numéricos se guardan en espacios de memoria, y esos espacios tienen un
nombre, un rótulo con el cual los podremos llamar y pedirle a la
computadora que los utilice para operar con ellos, los modifique, etc.
Ese nombre a cada espacio de memoria se asigna, al menos en Python, con
el símbolo = que significa de ahora en más: ``asignación''.

Pero no sólamente guardaremos valores numéricos. Además de haber
distintos tipos de valores numéricos, como veremos ahora, podemos
guardar otros tipos de datos, como texto (strings) y listas (lists)
entre muchos otros. Todos los tipos de valores que podremos almacenar
difieren entre si el espacio en memoria que ocupan y las operaciones que
podremos hacer con ellos.

Veamos un par de ejemplos

    \begin{Verbatim}[commandchars=\\\{\}]
{\color{incolor}In [{\color{incolor}2}]:} \PY{n}{x} \PY{o}{=} \PY{l+m+mi}{5}
        \PY{n}{y} \PY{o}{=} \PY{l+s}{\PYZsq{}}\PY{l+s}{Hola mundo!}\PY{l+s}{\PYZsq{}}
        \PY{n}{z} \PY{o}{=} \PY{p}{[}\PY{l+m+mi}{1}\PY{p}{,}\PY{l+m+mi}{2}\PY{p}{,}\PY{l+m+mi}{3}\PY{p}{]}
\end{Verbatim}

    Aquí hemos guardado en un espacio de memoria llamado por nosotros ``x''
la información de un valor de tipo entero, 5, en otro espacio de
memoria, que nosotros llamamos ``y'' guardamos el texto ``Hola mundo!''.
En Python, las comillas indican que lo que encerramos con ellas es un
texto. x no es un texto, así que Python lo tratará como variable para
manipular. ``z'' es el nombre del espacio de memoria donde se almacena
una lista con 3 elementos enteros.

Podemos hacer cosas con esta información. Python es un lenguaje
interpretado (a diferencia de otros como Java o C++), eso significa que
ni bien nosotros le pedimos algo a Python, éste lo ejecuta. Así es que
podremos pedirle por ejemplo que imprima en pantalla el contenido en y,
el tipo de valor que es x (entero) entre otras cosas.

    \begin{Verbatim}[commandchars=\\\{\}]
{\color{incolor}In [{\color{incolor}3}]:} \PY{k}{print} \PY{n}{y}
        \PY{k}{print} \PY{n+nb}{type}\PY{p}{(}\PY{n}{x}\PY{p}{)}
        \PY{k}{print} \PY{n+nb}{type}\PY{p}{(}\PY{n}{y}\PY{p}{)}\PY{p}{,} \PY{n+nb}{type}\PY{p}{(}\PY{n}{z}\PY{p}{)}\PY{p}{,} \PY{n+nb}{len}\PY{p}{(}\PY{n}{z}\PY{p}{)}
\end{Verbatim}

    \begin{Verbatim}[commandchars=\\\{\}]
Hola mundo!
<type 'int'>
<type 'str'> <type 'list'> 3
    \end{Verbatim}

    Vamos a utilizar mucho la función \emph{type()} para entender con qué
tipo de variables estamos trabajando. \emph{type()} es una función
predeterminada por Python, y lo que hace es pedir como argumento (lo que
va entre los paréntesis) una variable y devuelve inmediatamente el tipo
de variable que es.

    \subsubsection{Ejercicio 1}\label{ejercicio-1}

En el siguiente bloque cree las variables ``dato1'' y ``dato2'' y guarde
en ellas los textos ``estoy programando'' y ``que emocion!''. Con la
función \emph{type()} averigue qué tipo de datos se almacena en esas
variables.

    \begin{Verbatim}[commandchars=\\\{\}]
{\color{incolor}In [{\color{incolor}2}]:} \PY{c}{\PYZsh{} Realice el ejercicio 1}
\end{Verbatim}

    Para las variables \emph{integers}(enteros) y \emph{floats} (flotantes)
podemos hacer las operaciones matemáticas usuales y esperables. Veamos
un poco las compatibilidades entre estos tipos de variables.

    \begin{Verbatim}[commandchars=\\\{\}]
{\color{incolor}In [{\color{incolor}3}]:} \PY{n}{a} \PY{o}{=} \PY{l+m+mi}{5}
        \PY{n}{b} \PY{o}{=} \PY{l+m+mi}{7}
        \PY{n}{c} \PY{o}{=} \PY{l+m+mf}{5.0}
        \PY{n}{d} \PY{o}{=} \PY{l+m+mf}{7.0}
        \PY{k}{print} \PY{n}{a}\PY{o}{+}\PY{n}{b}\PY{p}{,} \PY{n}{b}\PY{o}{+}\PY{n}{c}\PY{p}{,} \PY{n}{a}\PY{o}{*}\PY{n}{d}\PY{p}{,} \PY{n}{a}\PY{o}{/}\PY{n}{b}\PY{p}{,} \PY{n}{a}\PY{o}{/}\PY{n}{d}\PY{p}{,} \PY{n}{c}\PY{o}{*}\PY{o}{*}\PY{l+m+mi}{2}
\end{Verbatim}

    \begin{Verbatim}[commandchars=\\\{\}]
12 12.0 35.0 0 0.714285714286 25.0
    \end{Verbatim}

    \subsubsection{Ejercicio 2}\label{ejercicio-2}

Calcule el resultado de \[ \frac{(2+7.9)^2}{4^{7.4-3.14*9.81}-1} \] y
guárdelo en una variable

    \begin{Verbatim}[commandchars=\\\{\}]
{\color{incolor}In [{\color{incolor}4}]:} \PY{c}{\PYZsh{} Realice el ejercicio 2. El resultado esperado es \PYZhy{}98.01}
\end{Verbatim}

    \subsection{Listas}\label{listas}

    Las listas son cadenas de datos de cualquier tipo, unidos por estar en
una misma variable, con posiciones dentro de esa lista, con las cuales
nosotros podemos llamarlas. En Python, las listas se enumeran desde el 0
en adelante.

Estas listas también tienen algunas operaciones que le son válidas.

Distintas son las tuplas. Las listas son editables, pero las tuplas no.
Esto es importante cuando, a lo largo del desarrollo de un código donde
necesitamos que ciertas cosas no cambien, no editemos por error valores
fundamentales de nuestro problema a resolver.

    \begin{Verbatim}[commandchars=\\\{\}]
{\color{incolor}In [{\color{incolor}5}]:} \PY{n}{lista1} \PY{o}{=} \PY{p}{[}\PY{l+m+mi}{1}\PY{p}{,} \PY{l+m+mi}{2}\PY{p}{,} \PY{l+s}{\PYZsq{}}\PY{l+s}{saraza}\PY{l+s}{\PYZsq{}}\PY{p}{]}
        \PY{k}{print} \PY{n}{lista1}\PY{p}{,} \PY{n+nb}{type}\PY{p}{(}\PY{n}{lista1}\PY{p}{)}
        \PY{k}{print} \PY{n}{lista1}\PY{p}{[}\PY{l+m+mi}{1}\PY{p}{]}\PY{p}{,} \PY{n+nb}{type}\PY{p}{(}\PY{n}{lista1}\PY{p}{[}\PY{l+m+mi}{1}\PY{p}{]}\PY{p}{)}
        \PY{k}{print} \PY{n}{lista1}\PY{p}{[}\PY{l+m+mi}{2}\PY{p}{]}\PY{p}{,} \PY{n+nb}{type}\PY{p}{(}\PY{n}{lista1}\PY{p}{[}\PY{l+m+mi}{2}\PY{p}{]}\PY{p}{)}
        \PY{k}{print} \PY{n}{lista1}\PY{p}{[}\PY{o}{\PYZhy{}}\PY{l+m+mi}{1}\PY{p}{]}
\end{Verbatim}

    \begin{Verbatim}[commandchars=\\\{\}]
[1, 2, 'saraza'] <type 'list'>
2 <type 'int'>
saraza <type 'str'>
saraza
    \end{Verbatim}

    \begin{Verbatim}[commandchars=\\\{\}]
{\color{incolor}In [{\color{incolor}6}]:} \PY{n}{lista2} \PY{o}{=} \PY{p}{[}\PY{l+m+mi}{2}\PY{p}{,}\PY{l+m+mi}{3}\PY{p}{,}\PY{l+m+mi}{4}\PY{p}{]}
        \PY{n}{lista3} \PY{o}{=} \PY{p}{[}\PY{l+m+mi}{5}\PY{p}{,}\PY{l+m+mi}{6}\PY{p}{,}\PY{l+m+mi}{7}\PY{p}{]}
        \PY{k}{print} \PY{n}{lista2}\PY{o}{+}\PY{n}{lista3}
        \PY{k}{print} \PY{n}{lista2}\PY{p}{[}\PY{l+m+mi}{2}\PY{p}{]}\PY{o}{+}\PY{n}{lista3}\PY{p}{[}\PY{l+m+mi}{0}\PY{p}{]}
\end{Verbatim}

    \begin{Verbatim}[commandchars=\\\{\}]
[2, 3, 4, 5, 6, 7]
9
    \end{Verbatim}

    \begin{Verbatim}[commandchars=\\\{\}]
{\color{incolor}In [{\color{incolor}7}]:} \PY{n}{tupla1} \PY{o}{=} \PY{p}{(}\PY{l+m+mi}{1}\PY{p}{,}\PY{l+m+mi}{2}\PY{p}{,}\PY{l+m+mi}{3}\PY{p}{)}
        \PY{n}{lista4} \PY{o}{=} \PY{p}{[}\PY{l+m+mi}{1}\PY{p}{,}\PY{l+m+mi}{2}\PY{p}{,}\PY{l+m+mi}{3}\PY{p}{]}
        \PY{n}{lista4}\PY{p}{[}\PY{l+m+mi}{2}\PY{p}{]} \PY{o}{=} \PY{l+m+mi}{0}
        \PY{k}{print} \PY{n}{lista4}
        \PY{c}{\PYZsh{}tupla1[0] = 0}
        \PY{k}{print} \PY{n}{tupla1}
\end{Verbatim}

    \begin{Verbatim}[commandchars=\\\{\}]
[1, 2, 0]
(1, 2, 3)
    \end{Verbatim}

    Hay formas muy cómodas de hacer listas. Presentamos una que utilizaremos
mucho, que es usando la función \emph{range}.

    \begin{Verbatim}[commandchars=\\\{\}]
{\color{incolor}In [{\color{incolor}8}]:} \PY{n}{listilla} \PY{o}{=} \PY{n+nb}{range}\PY{p}{(}\PY{l+m+mi}{10}\PY{p}{)}
        \PY{k}{print} \PY{n}{listilla}\PY{p}{,} \PY{n+nb}{type}\PY{p}{(}\PY{n}{listilla}\PY{p}{)}
\end{Verbatim}

    \begin{Verbatim}[commandchars=\\\{\}]
[0, 1, 2, 3, 4, 5, 6, 7, 8, 9] <type 'list'>
    \end{Verbatim}

    \subsubsection{Ejercicio 3}\label{ejercicio-3}

\begin{enumerate}
\def\labelenumi{\arabic{enumi}.}
\itemsep1pt\parskip0pt\parsep0pt
\item
  Haga una lista con los resultados de los últimos dos ejercicios y que
  la imprima en pantalla
\end{enumerate}

\begin{itemize}
\itemsep1pt\parskip0pt\parsep0pt
\item
  Sobreescriba en la misma variable la misma lista pero con sus
  elementos permutados e imprima nuevamente la lista
\end{itemize}

Ejemplo de lo que debería mostrarse en pantalla

\emph{{[}`estoy programando', `que emocion!', -98.01{]}}

\emph{{[}`estoy programando', -98.01, `que emocion!'{]}}

    \begin{Verbatim}[commandchars=\\\{\}]
{\color{incolor}In [{\color{incolor}9}]:} \PY{c}{\PYZsh{} Realice el ejercicio 3}
\end{Verbatim}

    \subsubsection{Ejercicio 4}\label{ejercicio-4}

\begin{enumerate}
\def\labelenumi{\arabic{enumi}.}
\itemsep1pt\parskip0pt\parsep0pt
\item
  Haga una lista con la función \emph{range} de 15 elementos y sume los
  elementos 5, 10 y 12
\end{enumerate}

\begin{itemize}
\item
  Con la misma lista, haga el producto de los primeros 4 elementos de
  esa lista
\item
  Con la misma lista, reste el último valor con el primero
\end{itemize}

    \begin{Verbatim}[commandchars=\\\{\}]
{\color{incolor}In [{\color{incolor} }]:} \PY{c}{\PYZsh{} Realice el ejercicio 4}
\end{Verbatim}

    \subsection{Booleans}\label{booleans}

    Este tipo de variable tiene sólo dos valores posibles: 1 y 0, o
\emph{True} y \emph{False}. Las utilizaremos escencialmente para que
Python reconozca relaciones entre números.

    \begin{Verbatim}[commandchars=\\\{\}]
{\color{incolor}In [{\color{incolor}10}]:} \PY{k}{print} \PY{l+m+mi}{5}\PY{o}{\PYZgt{}}\PY{l+m+mi}{4}
         \PY{k}{print} \PY{l+m+mi}{4}\PY{o}{\PYZgt{}}\PY{l+m+mi}{5}
         \PY{k}{print} \PY{l+m+mi}{4}\PY{o}{==}\PY{l+m+mi}{5} \PY{c}{\PYZsh{}La igualdad matemática se escribe con doble ==}
         \PY{k}{print} \PY{l+m+mi}{4}\PY{o}{!=}\PY{l+m+mi}{5} \PY{c}{\PYZsh{}La desigualdad matemática se escribe con !=}
         \PY{k}{print} \PY{n+nb}{type}\PY{p}{(}\PY{l+m+mi}{4}\PY{o}{\PYZgt{}}\PY{l+m+mi}{5}\PY{p}{)}
\end{Verbatim}

    \begin{Verbatim}[commandchars=\\\{\}]
True
False
False
True
<type 'bool'>
    \end{Verbatim}

    \subsubsection{Ejercicio 5}\label{ejercicio-5}

Averigue el resultado de \emph{4!=5==1}. ¿Dónde pondría paréntesis para
que el resultado fuera distinto?

    \begin{Verbatim}[commandchars=\\\{\}]
{\color{incolor}In [{\color{incolor}15}]:} \PY{c}{\PYZsh{} Realice el ejercicio 5}
\end{Verbatim}

    \subsection{Control de flujo: condicionales e iteraciones (if y for para
los
amigos)}\label{control-de-flujo-condicionales-e-iteraciones-if-y-for-para-los-amigos}

    Si en el fondo un programa es una serie de algoritmos que la computadora
debe seguir, un conocimiento fundamental para programar es saber cómo
pedirle a una computadora que haga operaciones si se cumple una
condición y que haga otras si no se cumple. Nos va a permitir hacer
programas mucho más complejos. Veamos entonces como aplicar un
\emph{if}.

    \begin{Verbatim}[commandchars=\\\{\}]
{\color{incolor}In [{\color{incolor}13}]:} \PY{n}{parametro} \PY{o}{=} \PY{l+m+mi}{5}
         \PY{k}{if} \PY{n}{parametro} \PY{o}{\PYZgt{}} \PY{l+m+mi}{0}\PY{p}{:} \PY{c}{\PYZsh{} un if inaugura un nuevo bloque indentado}
             \PY{k}{print} \PY{l+s}{\PYZsq{}}\PY{l+s}{Tu parametro es}\PY{l+s}{\PYZsq{}}\PY{p}{,} \PY{n}{parametro}\PY{p}{,} \PY{l+s}{\PYZsq{}}\PY{l+s}{y es mayor que cero}\PY{l+s}{\PYZsq{}}
             \PY{k}{print} \PY{l+s}{\PYZsq{}}\PY{l+s}{Gracias}\PY{l+s}{\PYZsq{}}
         \PY{k}{else}\PY{p}{:}            \PY{c}{\PYZsh{} el else inaugura otro bloque indentado}
             \PY{k}{print} \PY{l+s}{\PYZsq{}}\PY{l+s}{Tu parametro es}\PY{l+s}{\PYZsq{}}\PY{p}{,} \PY{n}{parametro}\PY{p}{,} \PY{l+s}{\PYZsq{}}\PY{l+s}{y es menor o igual que cero}\PY{l+s}{\PYZsq{}}
             \PY{k}{print} \PY{l+s}{\PYZsq{}}\PY{l+s}{Gracias}\PY{l+s}{\PYZsq{}}
         \PY{k}{print} \PY{l+s}{\PYZsq{}}\PY{l+s}{Vuelva pronto}\PY{l+s}{\PYZsq{}}
         \PY{k}{print} \PY{l+s}{\PYZsq{}}\PY{l+s}{ }\PY{l+s}{\PYZsq{}}
\end{Verbatim}

    \begin{Verbatim}[commandchars=\\\{\}]
Tu parametro es 5 y es mayor que cero
Gracias
Vuelva pronto
    \end{Verbatim}

    \begin{Verbatim}[commandchars=\\\{\}]
{\color{incolor}In [{\color{incolor}14}]:} \PY{n}{parametro} \PY{o}{=} \PY{o}{\PYZhy{}}\PY{l+m+mi}{5}
         \PY{k}{if} \PY{n}{parametro} \PY{o}{\PYZgt{}} \PY{l+m+mi}{0}\PY{p}{:} \PY{c}{\PYZsh{} un if inaugura un nuevo bloque indentado}
             \PY{k}{print} \PY{l+s}{\PYZsq{}}\PY{l+s}{Tu parametro es}\PY{l+s}{\PYZsq{}}\PY{p}{,} \PY{n}{parametro}\PY{p}{,} \PY{l+s}{\PYZsq{}}\PY{l+s}{y es mayor que cero}\PY{l+s}{\PYZsq{}}
             \PY{k}{print} \PY{l+s}{\PYZsq{}}\PY{l+s}{Gracias}\PY{l+s}{\PYZsq{}}
         \PY{k}{else}\PY{p}{:}            \PY{c}{\PYZsh{} el else inaugura otro bloque indentado}
             \PY{k}{print} \PY{l+s}{\PYZsq{}}\PY{l+s}{Tu parametro es}\PY{l+s}{\PYZsq{}}\PY{p}{,} \PY{n}{parametro}\PY{p}{,} \PY{l+s}{\PYZsq{}}\PY{l+s}{y es menor o igual que cero}\PY{l+s}{\PYZsq{}}
             \PY{k}{print} \PY{l+s}{\PYZsq{}}\PY{l+s}{Gracias}\PY{l+s}{\PYZsq{}}
         \PY{k}{print} \PY{l+s}{\PYZsq{}}\PY{l+s}{Vuelva pronto}\PY{l+s}{\PYZsq{}}
         \PY{k}{print} \PY{l+s}{\PYZsq{}}\PY{l+s}{ }\PY{l+s}{\PYZsq{}}
\end{Verbatim}

    \begin{Verbatim}[commandchars=\\\{\}]
Tu parametro es -5 y es menor o igual que cero
Gracias
Vuelva pronto
    \end{Verbatim}

    \subsubsection{Ejercicio 6}\label{ejercicio-6}

Haga un programa con un \emph{if} que imprima la suma de dos números si
un tercero es positivo, y que imprima la resta si el tercero es
negativo.

    \begin{Verbatim}[commandchars=\\\{\}]
{\color{incolor}In [{\color{incolor} }]:} \PY{c}{\PYZsh{} Realice el ejercicio 6}
\end{Verbatim}

    Para que Python repita una misma acción \emph{n} cantidad de veces,
utilizaremos la estructura \emph{for}. En cada paso, nosotros podemos
aprovechar el ``número de iteración'' como una variable. Eso nos servirá
en la mayoría de los casos.

    \begin{Verbatim}[commandchars=\\\{\}]
{\color{incolor}In [{\color{incolor}21}]:} \PY{n}{nueva\PYZus{}lista} \PY{o}{=} \PY{p}{[}\PY{l+s}{\PYZsq{}}\PY{l+s}{nada}\PY{l+s}{\PYZsq{}}\PY{p}{,}\PY{l+m+mi}{1}\PY{p}{,}\PY{l+m+mi}{2}\PY{p}{,}\PY{l+s}{\PYZsq{}}\PY{l+s}{tres}\PY{l+s}{\PYZsq{}}\PY{p}{,} \PY{l+s}{\PYZsq{}}\PY{l+s}{cuatro}\PY{l+s}{\PYZsq{}}\PY{p}{,} \PY{l+m+mi}{7}\PY{o}{\PYZhy{}}\PY{l+m+mi}{2}\PY{p}{,} \PY{l+m+mi}{2}\PY{o}{*}\PY{l+m+mi}{3}\PY{p}{,} \PY{l+m+mi}{7}\PY{o}{/}\PY{l+m+mi}{1}\PY{p}{,} \PY{l+m+mi}{2}\PY{o}{*}\PY{o}{*}\PY{l+m+mi}{3}\PY{p}{,} \PY{l+m+mi}{3}\PY{o}{*}\PY{o}{*}\PY{l+m+mi}{2}\PY{p}{]}
         \PY{k}{for} \PY{n}{i} \PY{o+ow}{in} \PY{n+nb}{range}\PY{p}{(}\PY{l+m+mi}{10}\PY{p}{)}\PY{p}{:}       \PY{c}{\PYZsh{} i es una variable que inventamos en el for, y que tomará los valores de la }
             \PY{k}{print} \PY{n}{nueva\PYZus{}lista}\PY{p}{[}\PY{n}{i}\PY{p}{]}   \PY{c}{\PYZsh{}lista que se genere con range(10)}
\end{Verbatim}

    \begin{Verbatim}[commandchars=\\\{\}]
nada
1
2
tres
cuatro
5
6
7
8
9
    \end{Verbatim}

    \subsubsection{Ejercicio 7}\label{ejercicio-7}

\begin{enumerate}
\def\labelenumi{\arabic{enumi}.}
\itemsep1pt\parskip0pt\parsep0pt
\item
  Haga otra lista con 16 elementos, y haga un programa que con un
  \emph{for} imprima solo los primeros 7
\end{enumerate}

\begin{itemize}
\itemsep1pt\parskip0pt\parsep0pt
\item
  Modifique el \emph{for} anterior y haga que imprima solo los elementos
  pares de su lista
\end{itemize}

    \begin{Verbatim}[commandchars=\\\{\}]
{\color{incolor}In [{\color{incolor} }]:} \PY{c}{\PYZsh{} Realice el ejercicio 7}
\end{Verbatim}

    La estructura \emph{while} es poco recomendada en Python pero es
importante saber que existe: consiste en repetir un paso mientras se
cumpla una condición. Es como un \emph{for} mezclado con un \emph{if}.

    \begin{Verbatim}[commandchars=\\\{\}]
{\color{incolor}In [{\color{incolor}22}]:} \PY{n}{i} \PY{o}{=} \PY{l+m+mi}{1}
         \PY{k}{while} \PY{n}{i} \PY{o}{\PYZlt{}} \PY{l+m+mi}{10}\PY{p}{:} \PY{c}{\PYZsh{} tener cuidado con los while que se cumplen siempre. Eso daría lugar a los loops infinitos.}
             \PY{n}{i} \PY{o}{=} \PY{n}{i}\PY{o}{+}\PY{l+m+mi}{1}
             \PY{k}{print} \PY{n}{i}
\end{Verbatim}

    \begin{Verbatim}[commandchars=\\\{\}]
2
3
4
5
6
7
8
9
10
    \end{Verbatim}

    \subsubsection{Ejercicio 8}\label{ejercicio-8}

\begin{enumerate}
\def\labelenumi{\arabic{enumi}.}
\itemsep1pt\parskip0pt\parsep0pt
\item
  Hacer una función que calcule el factorial de N, siendo N la única
  variable que recibe la función (Se puede pensar usando \emph{for} o
  usando \emph{while}).
\end{enumerate}

\begin{itemize}
\itemsep1pt\parskip0pt\parsep0pt
\item
  Hacer una función que calcule la sumatoria de los elementos de una
  lista.
\end{itemize}

    \begin{Verbatim}[commandchars=\\\{\}]
{\color{incolor}In [{\color{incolor} }]:} \PY{c}{\PYZsh{} Realice el ejercicio 8}
\end{Verbatim}

    \subsection{Funciones}\label{funciones}

    Pero si queremos definir nuestra propia manera de calcular algo, o si
queremos agrupar una serie de órdenes bajo un mismo nombre, podemos
definirnos nuestras propias funciones, pidiendo la cantidad de
argumentos que querramos.

Vamos a usar las funciones \emph{lambda} más que nada para funciones
matemáticas, aunque también tenga otros usos. Definamos el polinomio
$f(x) = x^2 - 5x + 6$ que tiene como raíces $x = 3$ y $x = 2$.

    \begin{Verbatim}[commandchars=\\\{\}]
{\color{incolor}In [{\color{incolor}16}]:} \PY{n}{f} \PY{o}{=} \PY{k}{lambda} \PY{n}{x}\PY{p}{:} \PY{n}{x}\PY{o}{*}\PY{o}{*}\PY{l+m+mi}{2} \PY{o}{\PYZhy{}} \PY{l+m+mi}{5}\PY{o}{*}\PY{n}{x} \PY{o}{+} \PY{l+m+mi}{6}
         \PY{k}{print} \PY{n}{f}\PY{p}{(}\PY{l+m+mi}{3}\PY{p}{)}\PY{p}{,} \PY{n}{f}\PY{p}{(}\PY{l+m+mi}{2}\PY{p}{)}\PY{p}{,} \PY{n}{f}\PY{p}{(}\PY{l+m+mi}{0}\PY{p}{)}
\end{Verbatim}

    \begin{Verbatim}[commandchars=\\\{\}]
0 0 6
    \end{Verbatim}

    Las otras funciones, las más generales, se las llama funciones
\emph{def}, y tienen la siguiente forma.

    \begin{Verbatim}[commandchars=\\\{\}]
{\color{incolor}In [{\color{incolor}17}]:} \PY{k}{def} \PY{n+nf}{promedio}\PY{p}{(}\PY{n}{a}\PY{p}{,}\PY{n}{b}\PY{p}{,}\PY{n}{c}\PY{p}{)}\PY{p}{:}
             \PY{n}{N} \PY{o}{=} \PY{n}{a} \PY{o}{+} \PY{n}{b} \PY{o}{+} \PY{n}{c} \PY{c}{\PYZsh{} Es importante que toda la función tenga su contenido indentado}
             \PY{n}{N} \PY{o}{=} \PY{n}{N}\PY{o}{/}\PY{l+m+mf}{3.0}
             \PY{k}{return} \PY{n}{N}
         \PY{n}{mipromedio} \PY{o}{=} \PY{n}{promedio}\PY{p}{(}\PY{l+m+mi}{5}\PY{p}{,}\PY{l+m+mi}{5}\PY{p}{,}\PY{l+m+mi}{7}\PY{p}{)} \PY{c}{\PYZsh{} Aquí rompimos la indentación}
         \PY{k}{print} \PY{n}{mipromedio}
\end{Verbatim}

    \begin{Verbatim}[commandchars=\\\{\}]
5.66666666667
    \end{Verbatim}

    \subsubsection{Ejercicio 9}\label{ejercicio-9}

Hacer una función que calcule el promedio de $n$ elementos dados en una
lista.

\textbf{Sugerencia}: utilizar las funciones \emph{len()} y \emph{sum()}
como auxiliares.

    \begin{Verbatim}[commandchars=\\\{\}]
{\color{incolor}In [{\color{incolor}18}]:} \PY{c}{\PYZsh{} Realice el ejercicio 9}
\end{Verbatim}

    \subsubsection{Ejercicio 10}\label{ejercicio-10}

Usando lo que ya sabemos de funciones matemáticas y las bifurcaciones
que puede generar un \emph{if}, hacer una función que reciba los
coeficientes $a, b, c$ de la parábola $f(x) = ax^2 + bx + c$ y calcule
las raíces \emph{si} son reales (es decir, usando el discriminante
$\Delta = b^2 - 4ac$ como criterio), y sino que imprima en pantalla una
advertencia de que el cálculo no se puede hacer en $\mathbb{R}$.

\subsubsection{Bonus track 1}\label{bonus-track-1}

Modificar la función anterior para que calcule las raíces de todos
modos, aunque sean complejas.

    \begin{Verbatim}[commandchars=\\\{\}]
{\color{incolor}In [{\color{incolor}24}]:} \PY{c}{\PYZsh{} Realice el ejercicio 10}
\end{Verbatim}

    \subsubsection{Ejercicio 11}\label{ejercicio-11}

\begin{enumerate}
\def\labelenumi{\arabic{enumi}.}
\itemsep1pt\parskip0pt\parsep0pt
\item
  Hacer una función que calcule el factorial de N, siendo N la única
  variable que recibe la función (Se puede pensar usando \emph{for} o
  usando \emph{while}).
\end{enumerate}

\begin{itemize}
\itemsep1pt\parskip0pt\parsep0pt
\item
  Hacer una función que calcule la sumatoria de los elementos de una
  lista.
\end{itemize}

    \begin{Verbatim}[commandchars=\\\{\}]
{\color{incolor}In [{\color{incolor} }]:} \PY{c}{\PYZsh{} Realice el ejercicio 11}
\end{Verbatim}

    \subsection{Bibliotecas}\label{bibliotecas}

    Pero las operaciones básicas de suma, resta, multiplicación y división
son todo lo que un lenguaje como Python puede hacer ``nativamente''. Una
potencia o un seno es álgebra no lineal, y para hacerlo, habría que
inventarse un algoritmo (una serie de pasos) para calcular por ejemplo
\emph{sen($\pi$)}. Pero alguien ya lo hizo, ya lo pensó, ya lo escribió
en lenguaje Python y ahora todos podemos usar ese algoritmo sin pensar
en él. Solamente hay que decirle a nuestro intérprete de Python dónde
está guardado ese algoritmo. \textbf{Esta posibilidad de usar algoritmos
de otros es fundamental en la programación, porque es lo que permite que
nuestro problema se limite solamente a entender cómo llamar a estos
algoritmos ya pensados y no tener que pensarlos cada vez}.

Vamos entonces a llamar a una \emph{biblioteca} llamada \emph{math} que
nos va a extender nuestras posibilididades matemáticas.

    \begin{Verbatim}[commandchars=\\\{\}]
{\color{incolor}In [{\color{incolor}14}]:} \PY{k+kn}{import} \PY{n+nn}{math} \PY{c}{\PYZsh{} Llamamos a una biblioteca}
         \PY{c}{\PYZsh{}Es usual usar math, en vez de m}
         \PY{n}{r1} \PY{o}{=} \PY{n}{math}\PY{o}{.}\PY{n}{pow}\PY{p}{(}\PY{l+m+mi}{2}\PY{p}{,}\PY{l+m+mi}{4}\PY{p}{)}
         \PY{n}{r2} \PY{o}{=} \PY{n}{math}\PY{o}{.}\PY{n}{cos}\PY{p}{(}\PY{n}{m}\PY{o}{.}\PY{n}{pi}\PY{p}{)}
         \PY{n}{r3} \PY{o}{=} \PY{n}{math}\PY{o}{.}\PY{n}{log}\PY{p}{(}\PY{l+m+mi}{100}\PY{p}{,}\PY{l+m+mi}{10}\PY{p}{)}
         \PY{n}{r4} \PY{o}{=} \PY{n}{math}\PY{o}{.}\PY{n}{log}\PY{p}{(}\PY{n}{m}\PY{o}{.}\PY{n}{e}\PY{p}{)}
         \PY{k}{print} \PY{n}{r1}\PY{p}{,} \PY{n}{r2}\PY{p}{,} \PY{n}{r3}\PY{p}{,} \PY{n}{r4}
\end{Verbatim}

    \begin{Verbatim}[commandchars=\\\{\}]
16.0 -1.0 2.0 1.0
    \end{Verbatim}

    Para entender cómo funcionan estas funciones, es importante recurrir a
su \emph{documentation}. La de esta biblioteca en particular se
encuentra en

https://docs.python.org/2/library/math.html

    \subsubsection{Ejercicio 12}\label{ejercicio-12}

Use Python como calculadora y halle los resultados de

\begin{enumerate}
\def\labelenumi{\arabic{enumi}.}
\itemsep1pt\parskip0pt\parsep0pt
\item
  $\log(\cos(2\pi))$
\end{enumerate}

\begin{itemize}
\itemsep1pt\parskip0pt\parsep0pt
\item
  $\text{atanh}(2^{\cos(e)} -1) $
\item
  $\sqrt{x^2+2x+1}$ con $x = 125$
\end{itemize}

    \begin{Verbatim}[commandchars=\\\{\}]
{\color{incolor}In [{\color{incolor}15}]:} \PY{c}{\PYZsh{} Realice el ejercicio 12}
\end{Verbatim}

    \subsubsection{Bonus track 2}\label{bonus-track-2}

Ahora que nos animamos a buscar nuevas bibliotecas y definir funciones,
buscar la función \emph{newton()} de la biblioteca
\textbf{scipy.optimize} para hallar $x$ tal que se cumpla la siguiente
ecuación no lineal \[\frac{1}{x} = ln(x)\]

    \subsection{Gráficos}\label{gruxe1ficos}

    Lo siguiente que Python tiene de interesante para usar son sus
facilidades para hacer gráficos. La biblioteca \textbf{matplotlib} nos
ayudará en este caso. También utilizaremos una biblioteca llamada
\textbf{numpy} por sus enormes ventajas para el cálculo numérico, motivo
de nuestro próximo taller. Primero, definimos un vector que nos hace de
dominio, luego, un vector imagen de alguna función, y luego haremos el
gráfico. Se muestran aquí algunas de las opciones que tiene matplotlib
para presentar un gráfico, pero yendo a la documentación podrán
encontrar infinidad de herramientas para hacer esto.

    \begin{Verbatim}[commandchars=\\\{\}]
{\color{incolor}In [{\color{incolor}25}]:} \PY{k+kn}{from} \PY{n+nn}{matplotlib} \PY{k+kn}{import} \PY{n}{pyplot} \PY{k}{as} \PY{n}{plt}
         \PY{k+kn}{import} \PY{n+nn}{numpy} \PY{k+kn}{as} \PY{n+nn}{np}
         \PY{o}{\PYZpc{}}\PY{k}{matplotlib} pyplot
         
         \PY{n}{x} \PY{o}{=} \PY{n}{np}\PY{o}{.}\PY{n}{linspace}\PY{p}{(}\PY{o}{\PYZhy{}}\PY{l+m+mi}{10}\PY{p}{,} \PY{l+m+mi}{10}\PY{p}{,} \PY{l+m+mi}{200}\PY{p}{)} \PY{c}{\PYZsh{} con la función linspace generaremos un vector con componentes equidistantes.}
         \PY{n}{y} \PY{o}{=} \PY{n}{x}\PY{o}{*}\PY{o}{*}\PY{l+m+mi}{2} \PY{c}{\PYZsh{} el vector imagen será igual de largo que x}
         \PY{n}{plt}\PY{o}{.}\PY{n}{plot}\PY{p}{(}\PY{n}{x}\PY{p}{,}\PY{n}{y}\PY{p}{,} \PY{l+s}{\PYZsq{}}\PY{l+s}{\PYZhy{}}\PY{l+s}{\PYZsq{}}\PY{p}{,} \PY{n}{color} \PY{o}{=} \PY{l+s}{\PYZsq{}}\PY{l+s}{red}\PY{l+s}{\PYZsq{}}\PY{p}{,} \PY{n}{label} \PY{o}{=} \PY{l+s}{\PYZsq{}}\PY{l+s}{Curva x**2}\PY{l+s}{\PYZsq{}}\PY{p}{)} \PY{c}{\PYZsh{} ver qué pasa con \PYZsq{}r\PYZsq{}, \PYZsq{}g\PYZsq{}, \PYZsq{}*\PYZsq{} entre otros}
         \PY{n}{plt}\PY{o}{.}\PY{n}{title}\PY{p}{(}\PY{l+s}{\PYZsq{}}\PY{l+s}{Mi primer ploteo}\PY{l+s}{\PYZsq{}}\PY{p}{)}
         \PY{n}{plt}\PY{o}{.}\PY{n}{xlabel}\PY{p}{(}\PY{l+s}{\PYZsq{}}\PY{l+s}{Eje de las x}\PY{l+s}{\PYZsq{}}\PY{p}{)}
         \PY{n}{plt}\PY{o}{.}\PY{n}{ylabel}\PY{p}{(}\PY{l+s}{\PYZsq{}}\PY{l+s}{Eje de las y}\PY{l+s}{\PYZsq{}}\PY{p}{)}
         \PY{c}{\PYZsh{}plt.xlim(\PYZhy{}5,5)}
         \PY{c}{\PYZsh{}plt.ylim(0,4)}
         \PY{n}{plt}\PY{o}{.}\PY{n}{legend}\PY{p}{(}\PY{n}{loc} \PY{o}{=} \PY{l+s}{\PYZsq{}}\PY{l+s}{best}\PY{l+s}{\PYZsq{}}\PY{p}{)}
         \PY{n}{grid}\PY{p}{(}\PY{n+nb+bp}{True}\PY{p}{)}
\end{Verbatim}

    \begin{center}
    \adjustimage{max size={0.9\linewidth}{0.9\paperheight}}{introduccion_files/introduccion_63_0.png}
    \end{center}
    { \hspace*{\fill} \\}
    
    \subsubsection{Ejercicio 13}\label{ejercicio-13}

Hallar de manera gráfica la $x$ que resuelve la ecuación del primer
bonus track. Recordamos la ecuación. \[ \frac{1}{x}=ln(x)\]

    \begin{Verbatim}[commandchars=\\\{\}]
{\color{incolor}In [{\color{incolor}26}]:} \PY{c}{\PYZsh{} Bloque para el ejercicio 13}
\end{Verbatim}

    \section{La importancia de las
referencias}\label{la-importancia-de-las-referencias}

    Para más referencias pueden googlear. Dejamos algunas de referencia:

http://pybonacci.org/2012/06/07/algebra-lineal-en-python-con-numpy-i-operaciones-basicas/

http://relopezbriega.github.io/blog/2015/06/14/algebra-lineal-con-python/

http://pendientedemigracion.ucm.es/info/aocg/python/modulos\_cientificos/numpy/index.html

Pero es importantísimo manejarse con la documentación de las bibliotecas
que se utilizan

https://docs.python.org/2/library/math.html

http://docs.scipy.org/doc/numpy/reference/routines.linalg.html

http://matplotlib.org/api/pyplot\_api.html

    \section{Recursos}\label{recursos}

Para seguir profundizando con la programación en Python, ofrecemos
distintos recursos

Un tutorial: http://www.learnpython.org/

\emph{How to think like a computer scientist} (aprendizaje interactivo):
http://interactivepython.org/runestone/static/thinkcspy/index.html

Otro tutorial, en inglés, pero muy completo:
http://learnpythonthehardway.org/book

Coursera, que nunca está de más:
https://www.coursera.org/learn/interactive-python-1

Otro más: https://es.coursera.org/learn/python

Y por fuera del taller, seguimos en contacto. Tenemos un grupo de
Facebook donde pueden hacerse consultas y otros chicos que fueron al
taller antes o aprendieron por sus medios podrán responderles. El grupo
es https://www.facebook.com/groups/303815376436624/?fref=ts

    \section{Y para seguir manijeando}\label{y-para-seguir-manijeando}

Tenemos otro taller dentro de una semana donde profundizaremos el uso de
algunas herramientas clave para el análisis de datos de Laboratorio. ¡No
te lo podés perder!

A parte, en nuestro Github
(https://github.com/fifabsas/talleresfifabsas) iremos colgando nuevo
material para ejemplificar, con problemas de las materias de Física
resueltos numéricamente, herramientas de Labo y mucho más.

    \section{Agradecimientos}\label{agradecimientos}

Todo esto es posible gracias al aporte de mucha gente.

\begin{itemize}
\itemsep1pt\parskip0pt\parsep0pt
\item
  Gente muy copada del DF como Hernán Grecco, Guillermo Frank, Osvaldo
  Santillán , Martín Elías Costa y Agustín Corbat por hacer aportes a
  estos talleres de diferentes maneras, desde poner su apellido para que
  nos presten un labo hasta venir como invitado a un taller.
\item
  El Departamento de Computación que cuatrimestre a cuatrimestre nos
  presta los labos desinteresadamente.
\item
  Los estudiantes del CODEP de Física, el CECEN y mucha gente que ayuda
  con la difusión.
\item
  Pibes de la FIFA que prestan su tiempo a organizar el material y
  llevan a cabo el taller.
\item
  Todos los que se acercan y piden que estos talleres se sigan dando y
  nos siguen llenando los Labos. Sí ¡Gracias a todos ustedes!
\end{itemize}


    % Add a bibliography block to the postdoc
    
    
    
    \end{document}
