\documentclass[twoside]{article}
%\include{amsfonts}
\usepackage{amssymb,amsmath,amsthm,latexsym,epsfig,euscript,multicol}
%\usepackage{enumitem}
\usepackage{graphicx}
\usepackage{float}

\usepackage{hyperref}

\usepackage[utf8]{inputenc}
\usepackage[spanish]{babel}% idioma castellano

% Caracteres especiales
\def\A{\mathbb{A}}
\def\C{\mathbb{C}}
\def \N{\mathbb{N}}
\def \P{\mathbb{P}}
\def \Q{\mathbb{Q}}
\def \R{\mathbb{R}}
\def \Z{\mathbb{Z}}

\def\zC{\mathbb{C}}
\def \zN{\mathbb{N}}

\def \zQ{\mathbb{Q}}
\def \zR{\mathbb{R}}
\def \zZ{\mathbb{Z}}

%  Ejercicio:
\newtheorem{ejer}{Unidad}
\newcommand{\bej}{\begin{ejer} \rm}
\newcommand{\fej}{\end{ejer}}


%
\def\d{\displaystyle}

%Encabezado y pie de página
\usepackage{fancyhdr} % Headers and footers
\setlength{\headwidth}{16.5cm}
\pagestyle{fancy} % All pages have headers and footers


\renewcommand{\headrulewidth}{1.5pt}


\renewcommand{\footrulewidth}{1.5pt}

\fancyhead{} % Blank out the default header
\fancyfoot{} % Blank out the default footer
\fancyhead[L]{Taller de Programación para laboratorio} % Custom header text
\fancyhead[R]{{\small Departamento de Física, FCEN, UBA}}% Custom header text
\fancyfoot[RO,LE]{\thepage} % Custom footer text
\fancyfoot[LO,RE]{FIFA}


\topmargin-2cm \vsize 29.5cm \hsize 21cm
\setlength{\textwidth}{16.5cm} \setlength{\textheight}{23.5cm}
\setlength{\oddsidemargin}{0.0cm}
\setlength{\evensidemargin}{0.0cm}

%\linespread{1.4cm}
\usepackage{setspace}
 \onehalfspacing

%-------------------------------------------------------------------------------------------------
%-------------------------------COMIENZA EL TEXTO-------------------------------------------------
%-------------------------------------------------------------------------------------------------
\begin{document}
\thispagestyle{empty}

\vskip 1cm

\centerline{{\small \textit{Departamento de Física}}}
\centerline{{\small \textit{Facultad de Ciencias Exactas y Naturales }}}
\centerline{{\small \textit{Universidad de Buenos Aires}}}
\vskip 1cm

\centerline{{\bf\large {\sc Programación para laboratorio}}}

\centerline{{\bf\large {\sc $1^{er}$ cuatrimestre 2015}}}

\centerline{{\ttfamily Talleres FIFA}}

\begin{figure}[H]
 \centering
 \includegraphics[width=0.15\textwidth]{fig/Python.png}
 \label{FIFA}
 \end{figure}
 

\bigskip


\centerline{\bf  Ajuste lineal}

\bigskip

Se puede probar, dentro del contexto de \textit{cuadrados mínimos} que para un ajuste lineal vale que:

\begin{equation}
a = \displaystyle\frac{N\sum x_i y_i - \sum x_i \sum y_i}{N \sum x_i^2 - (\sum x_i)^2}
\end{equation}

\begin{equation}
b = \d\frac{\sum x_i^2 \sum y_i - \sum x_i \sum x_i y_i}{N \sum x_i^2 - (\sum x_i)^2}
\end{equation}


Donde las desviaciones son calculadas como:

\begin{equation}
\sigma_a = \sigma_y \sqrt{ \d\frac{N}{N \sum x_i^2 - (\sum x_i)^2}}
\end{equation}

\begin{equation}
\sigma_b = \sigma_y \sqrt{\d\frac{\sum x_i^2}{N \sum x_i^2 - (\sum x_i)^2}}
\end{equation}

\begin{equation}
\sigma_y = \sqrt{\d\frac{1}{N-2}\sum [y_i -y(x_i)]^2}
\end{equation}



En último lugar, la determinación del coeficiente de Pearson, R2, para la bondad del ajuste se calcula, a partir de los datos experimentales, por medio de la siguiente expresión:

\begin{equation}
R^2 = \d\frac{[N\sum x_i y_i - \sum x_i \sum y_i]^2}{[N \sum x_i^2 - (\sum x_i)^2][N \sum y_i^2 - (\sum y_i)^2]}
\end{equation}

Aplicaremos entonces estas funciones considerando que en \textit{Python}, estas expresiones se escriben como:

\begin{enumerate}
\item $x_i$ es x
\item $x_i^2$ es x**2
\item $\sqrt{x_i}$ es np.sqrt(x)
\item $\sum x_i$ es np.sum(x)
\item $\sum x_i^2$ es np.sum(x**2)
\item $\sum x_i y_i$ es np.sum(x*y)
\item $(\sum x_i)^2$ es np.sum(x)**2
\end{enumerate}

\end{document}
